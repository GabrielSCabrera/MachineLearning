\section{Introduction}
    In the field of statistics, regression methods have been in development for many decades; more recently, computational developments have managed to implement these methods in new ways, allowing us to perform exceptional feats unseen in the past. An \textit{Artificial Neural Network} (ANN) is one such exceptional statistical tool – such a network is an extension of simple linear regression, expanding to have greater dynamic and customized prediction power. ANNs have recently been applied to a phenomenal amount of research in practically \textit{all} fields; examples of this are ANNs with the ability to identify and distinguish between faces (see the work by \textit{H.A. Rowley}, \textit{S. Baluja}, and \textit{T. Kanade} on face detection using neural networks \cite{NNWfacedetection}), or networks with the ability to predict localized solar radiation to optimize solar panel design (see the work of \textit{A. Mellit}, \textit{M. Menghanem}, and \textit{M. Bendekhis} on sunshine duration and air temperature predictions using neural networks \cite{NNWsunshine}). \\\\
    The project presented aims to implement such statistical methods (ANNs) in an attempt to predict the behaviour of \textit{credit card holders}. More specifically, we are interested in the probability that a given customer will default on his/her credit card debt. For this, a large credit card holder dataset from a Taiwanese bank is used for model training and testing. This dataset has been previously analyzed by \textit{I-Cheng Yeh et al.} \cite{CCdata}, offering us a good performance benchmark for the methods derived and implemented in the study presented. \\\\
    Various methods are also applied to this data for comparison: firstly, a method of \textit{stochastic gradient descent} (with mini-batches) is applied to the data. This model provides a good basis of accuracy as it is relatively simple to implement in relation to a neural network. A \textit{multilayer perceptron} model is subsequently built for comparison. This model is the principal type of ANN studied in this paper, as it is a reasonable tradeoff between complexity and performance in the context ANNs. This network is first built around predicting the credit card data, or a so-called \textit{binary classification} case, though the model is later applied to a more general \textit{regression} case. Data is generated for the regression case using \textit{Franke's function}, and the ANN will be compared to the so-called \textit{Ridge} regression scheme. The ridge regression scheme is quite simple and effective, once again providing a baseline which the ANN is expected to outperform.\\\\
    The report is split into sections designed to raise subjects for discussion and lead the reader to similar conclusions as those presented. A section on \textit{Theory and Algorithms} provides the mathematical background needed for the implementation of the regression schemes presented (and implemented) in the \textit{Method} section. Following this are the \textit{Results}, where the performances of the implemented algorithms are demonstrated and subsequently discussed in the \textit{Discussion} section. The paper is then summarized in a \textit{Conclusion} of the research.\\\\
    This project is a collaboration between \textit{Steinn H. Magnússon}, \textit{Gabriel S. Cabrera}, and \textit{Bendik S. Dalen}. The code used to produce the results of the study can be found on the following \href{https://github.com/GabrielSCabrera/MachineLearning/tree/master/FYS-STK4155/Project_2/Gabriel/project}{\textbf{Project 2 Github Repository}}. 