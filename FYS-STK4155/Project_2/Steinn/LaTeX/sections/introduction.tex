\section{Introduction}
    Regression methods have been in development in the statistical context for many decades, though the recent computational revolution has applied these methods to do exceptional things that no statistician at the time likely expected. An \textit{Artificial Neural Network} (ANN) is one such exceptional statistical tool. Such a network is an extension of simple linear regression, expanding to have greater dynamic and customized prediction power. Neural networks have recently been applied to a phenomenal amount of research in practically all fields; examples of this are neural networks with the ability to detect faces and distinguish between them (see to work by H.A. Rowley, S. Baluja, T. Kanade on face detection using neural networks \cite{NNWfacedetection}), and networks with the ability to predict the solar radiation in a given area for efficient solar paneling design (see to work by A. Mellit, M. Menghanem, M. Bendekhis on sunshine duration and air temperature predictions using neural networks \cite{NNWsunshine}). \\\\
    The project presented aims to use such statistical methods (ANNs) in an attempt to predict the behaviour of credit card holders. More specifically, what the probability is that any given customer will choose to default on his/her credit card. For this, a large credit card holder data set from a bank in Taiwan is used for model training and testing. This data has been analysed previously by I-Cheng Yeh et al. \cite{CCdata}, something which offers a good performance benchmark for the methods derived and implemented in the study presented. \\\\
    Various methods are applied to this data for comparison. Firstly, a method of stochastic gradient descent with mini-batches is applied to the data. This model provides a good basis of accuracy as it is a somewhat simple scheme to implement in relation to the neural network. A \textit{multilayer perceptron} model is subsequently built for comparison. This model is the principal type of ANN studied in this paper, as it is a decent tradeoff between complexity and performance in a neural network context. 
    This network is first built around predicting the credit card data, or a so-called \textit{binary classification} case. The model is later applied to a more general \textit{regression} case. Data is generated for the regression case using \textit{Franke's Function}, and the ANN will be compared to the so-called \textit{Ridge} regression scheme. The ridge regression scheme is quite simple and effective, once again providing a baseline which the ANN is expected to outperform.\\\\
    The report is split into sections designed to raise subjects for discussion and lead the reader to similar conclusions to those presented. A section on \textit{Theory and Algorithms} provides the mathematical background needed for the implementation of the regression schemes presented and implemented in the \textit{Method} section. Following the method section are the \textit{Results}, where the performances of the implemented algorithms are demonstrated and subsequently discussed in the \textit{Discussion} section. The paper is then summarized in a \textit{Conclusion} of the research.\\\\
    This project is a collaboration between Steinn H. Magnússon, Gabriel S. Cabrera, and Bendik S. Dalen. The code used to produce the results of the study can be found on the following \href{https://github.com/GabrielSCabrera/MachineLearning/tree/master/FYS-STK4155/Project_2/Gabriel/project}{\textbf{Project 2 Github Repository}}. 