\documentclass[twoside,twocolumn,10pt]{revtex4-1}
\usepackage[utf8]{inputenc}
\usepackage[left=1.4cm, right=1.4cm, top=1.9cm, bottom=1.9cm]{geometry}
\usepackage{float}
\setlength{\parindent}{0mm}
\setlength{\parskip}{1.5mm}
\usepackage{amsmath}
\usepackage{physics}
\usepackage{graphicx}
\usepackage{fancyhdr}
\usepackage{ dsfont }
\usepackage{ amssymb }
\usepackage{ wasysym }
\usepackage{hyperref}
\usepackage{lipsum}
\usepackage{colortbl}
\usepackage{forest}
\usepackage{booktabs}
\usepackage{lastpage}
\usepackage{datetime}
\usepackage{csquotes}
\usepackage{outlines}
\usepackage{multirow}
\usepackage{color}

\usepackage{textcomp}
\definecolor{listinggray}{gray}{0.9}
\definecolor{lbcolor}{rgb}{0.9,0.9,0.9}

\usepackage{listings}
\lstset{
	backgroundcolor=\color{lbcolor},
	tabsize=4,
	rulecolor=,
	language=python,
        basicstyle=\scriptsize,
        upquote=true,
        aboveskip={1.5\baselineskip},
        columns=fixed,
	numbers=left,
        showstringspaces=false,
        extendedchars=true,
        breaklines=true,
        prebreak = \raisebox{0ex}[0ex][0ex]{\ensuremath{\hookleftarrow}},
        frame=single,
        showtabs=false,
        showspaces=false,
        showstringspaces=false,
        identifierstyle=\ttfamily,
        keywordstyle=\color[rgb]{0,0,1},
        commentstyle=\color[rgb]{0.133,0.545,0.133},
        stringstyle=\color[rgb]{0.627,0.126,0.941}
        }

% Table rules
\newcolumntype{C}{>{$}c<{$}}
\AtBeginDocument{
	\heavyrulewidth=.08em
	\lightrulewidth=.05em
	\cmidrulewidth=.03em
	\belowrulesep=.65ex
	\belowbottomsep=0pt
	\aboverulesep=.4ex
	\abovetopsep=0pt
	\cmidrulesep=\doublerulesep
	\cmidrulekern=.5em
	\defaultaddspace=.5em
}

\pagestyle{fancy}

% Allows the user to create sub-bullet points
\newcommand{\SubItem}[1]{
    {\setlength\itemindent{15pt} \item[-] #1}
}

\begin{document}
	
	\title{STK-IN4300 Compendium}
	\author{Gabriel Sigurd Cabrera}
	\date{\today}
	
	% Define \rightmark (to state III. Method f.ex) for the Left header
	
%	\renewcommand{\sectionmark}[1]{\markright{\MakeUppercase{\roman{section}}.\ #1}}
%	\renewcommand{\chaptermark}[1]{\markboth{#1}{}}
	
	% Define the date and year for Right header
	\newdateformat{monthyeardate}{%
		\monthname[\THEMONTH] \THEYEAR}
	
	% Headers
	\fancyhf{} 
	\fancyhead[L]{\rightmark}
	\fancyhead[R]{University of Oslo, \monthyeardate\today}
	\fancyfoot[C]{\thepage}
	
	\maketitle
	\twocolumngrid
	
	\section{Overview of Topics}
	
	\subsection{Lecture 1}	
	
	\subsubsection{Basics}
	
	Typical Scenario:
	
	An \textit{outcome} $Y$ (\textit{dependent variable}, \textit{response}) can be \textit{categorical} or \textit{qualitative}.
	
	We want to predict this outcome based on a set of \textit{features} $X_1, X_2, ..., X_p$ (\textit{independent variables}, \textit{predictors}).
	
	In practice, we have a \textit{training set} that is used to create a \textit{learner} (or model/rule $f(X_i) \approx Y_i$.)
	
	A \textit{supervised learning problem} is when the outcome is measured in the training data, and can be used to construct a learner $Y$.
	
	\subsection{Least Squares Estimate}	
	
	Given a training set $\lbrace (x_ {i1}, x_{i2}, ..., x_{ip}, y_i \rbrace$ a \textit{least-squares} model is given by:
	
	\begin{equation*}
	y_i = \beta_0 + \sum_{j=1}^p \beta_j x_{ij} + \epsilon_i
	\end{equation*}
	
	With a least-square estimate:
	
	\begin{equation*}
	\hat{\beta} = (X^\textsc{t} X)^{-1} X^\textsc{t} y
	\end{equation*}	
	
	Where $X^\textsc{t}X$ is known as the \textit{Gramian}.
	
	\subsection{Invertability}
	
	If $X^\textsc{t} X$ is \textit{not invertible}, then we can use \textit{dimension reduction} or \textit{shrinkage methods}.
	
	Some dimension reduction methods are to:
	
	\begin{itemize}
	\item Remove variables with \textit{low correlation} (forward selection/back substitution)
	\item More formal subset selection
	\item Selecting optimal linear combinations of variables (\textit{principal component analysis}.)
	\end{itemize}

	Some shrinkage methods are:
	
	\begin{itemize}
	\item \textit{Ridge regression}
	\item \textit{LASSO}
	\item \textit{Elastic net}
	\end{itemize}
	
	\subsection{Conventions}
	
	\textit{Quantitative response}: \textit{Regression}
	
	\textit{Qualitative response}: \textit{Classification}
	
	\subsection{Least-Squares}
	
	For \textit{ordinary least-squares} (OLS), we estimate $\beta$ by minimizing the \textit{residual sum of squares} (RSS):
	
	\begin{equation*}
	\text{RSS} (\beta) = \sum_{i=1}^N (y_i - x_i^\textsc{t} \beta)^2 = (y - X \beta)^\textsc{t} (y - X \beta)
	\end{equation*}
	
	Where $X \in \mathbb{R}^{N \times p}$, $X \in \mathbb{R}^N$.
	
	\subsection{K-Nearest-Neighbors}
	
	The \textit{k-nearest-neighbors} (KNN) of $x$ is the mean:
	
	\begin{equation*}
	\hat{Y}(x) = \frac{1}{k} \sum_{i:x_i \in N_k(x)} y_i
	\end{equation*}
	
	\subsection{Other Methods}
	
	OLS and KNN are the basis of most modern techniques; some of these are:
	
	\begin{itemize}
	\item \textit{Kernel methods} that weigh data according to distance
	\item In higher dimensions, weighing variables based on correlation
	\item Local regression models
	\item Linear models of functions of $X$
	\item \textit{Projection pursuit} and \textit{neural network}
	\end{itemize}
	
	\subsection{Statistical Decision Theory}
	
	\textit{Statistical decision theory} gives a \textit{mathematical framework} for finding the optimal learner.
	
	Given $X \in \mathbb{R}^p$, $Y \in \mathbb{R}$ and a \textit{joint distribution} $p(X,Y)$, our goal is to find a function $f(X)$ for predicting $Y$ given $X$.
	
	This requires a \textit{loss function} $L(Y,f(X))$ for penalizing errors in $f(X)$ when the truth is $Y$.
	
	An example is \textit{squared error loss}:
	
	\begin{equation*}
	L(Y,f(X) = (Y - f(X))^2
	\end{equation*}	 
	
	The \textit{expected prediction error} of $f(X)$ is given by:
	
	\begin{equation*}
	\text{EPE}(f) = E_{X,Y} [L(Y,f(X)] = \int_{x,y} L(y,f(x)) p(x,y) \ \text{d}x \ \text{d}y
	\end{equation*}
	
	Next, we must find the $f$ that minimizes $\text{EPE}(f)$.
	
	For the \textit{squared error loss} $L(Y,f(X) = (Y - f(X))^2$, we have:
	
	\begin{equation*}
	\text{EPE}(f) = E_{X,Y} [(Y - f(X))^2] = E_X E_{Y|X} [(Y - f(X))^2 | X] 
	\end{equation*}
	
	It is sufficient to minimize $E_{Y|X} [(Y - f(X))^2 | X]$:
	
	\begin{equation*}
	f(x) = \text{argmin}_c E_{Y|X} [(Y - c)^2 | X = x] = E[Y|X=x]
	\end{equation*}
	
	This is known as the \textit{conditional expectation}, or the \textit{regression function}.  This implies that the best prediction of $Y$ at any point $X = x$ is the \textit{conditional mean}.
	
	\subsection{Error Decomposition}
	
	$E[(Y - \hat{f}(X))^2] = \underbrace{\sigma^2}_{\text{irreducible error}} + \underbrace{\underbrace{\text{Var}(\hat{f}(X))}_{\text{variance}} + \underbrace{E[\hat{f}(X) - f(X)]^2}_{\text{bias}^2}}_{MSE}$
	
	
	\subsection{Assumptions for OLS}
	
	\begin{itemize}
	\item A function is linear in its arguments; $f(x) \approx x^\textsc{t} \beta$.
	\item $\text{argmin}_\beta E [(Y - X^\textsc{t}\beta)^2 | X = x] \to \beta = E[XX^\textsc{t}]^{-1} E[XY]$.
	\item Replacing the expectations by averages over the training data leads to $\hat{\beta}$.
	\end{itemize}
	
	\subsection{Assumptions for KNN}
	
	\begin{itemize}
	\item Uses $f(x) = E[Y|X = x]$ directly.
	\item $\hat{f}(x_i) = \text{mean}(y_i)$ for observed $x_i$.
	\item Normally, there is at most one observation for each point $x_i$.
	\item Uses points in the neighborhood:
	
		\begin{equation*}
		\hat{f}(x) = \text{mean}(y_i | x_i \in N_k(x))
		\end{equation*}

	\item There are two approximations:
	\SubItem{\textit{Expectation} is approximated by averaging over sample data.}
	\SubItem{\textit{Conditioning} on a point is related to conditioning on a neighborhood.}
	\item $f(x)$ can be approximated by a \textit{locally constant function}.
	\item For $N \to \infty$, all $x_i \in N_k(x) \approx x$.
	\item For $k \to \infty$, $\hat{f}(x)$ is getting more stable.
	\item Under mild regularity conditions on $p(X,Y)$:
		\begin{equation*}
		\hat{f}(x) \to E[Y|X = x] \ \text{for} \ N, k \to \infty \ \text{s.t.} \ k/N \to 0
		\end{equation*}		
	\item It is unnecessary to implement the \textit{squared loss error} function ($L_2$ loss function.)
	\item A valid alternative is the $L_1$ loss function, whose solution is the conditional median:
		\begin{equation}
		\hat{f}(x) = \text{median}(Y|X=x)
		\end{equation}
	\item More robust estimates than those obtained with conditional mean.
	\item The $L_1$ loss function has discontinuities in its derivatives which leads to numerical difficulties.
	\end{itemize}
	
	\subsection{Conclusion}
	
	OLS: stable but biased
	
	KNN: less biased and less stable
	
	For higher dimensions, KNN suffers from the \textit{curse of dimensionality}
	
	\section{Lecture 2}
	
	\subsection{Gauss-Markov Theorem}
	
	The least square estimator $\hat{\theta} = a^\textsc{t}(X^\textsc{t}X)^{-1}X^\textsc{t}y$ is the:
	
	\begin{tabular}{l l c}
	\textbf{B} & est & smallest error (MSE) \\
	\textbf{L} & inear & $\hat{\theta} = a^\textsc{t} \beta$ \\
	\textbf{U} & nbiased & $E[\hat{\theta}] = \theta$ \\
	\textbf{E} & stimator &
	\end{tabular}
	
	Given the \textit{error decomposition}, then any estimator $\tilde{\theta} = c^\textsc{t}Y$ s.t. $E[c^\textsc{t}Y] = a^\textsc{t} \hat{\beta}$ has $\text{Var}(c^\textsc{t}Y) \geq \text{Var}(a^\textsc{t}\hat{\beta})$.
	
	\subsection{Hypothesis Testing}
	
	To test $H_0 : \beta_j = 0$ we use the \textit{Z-score statistic}:
	
	\begin{equation}
	z_j = \frac{\hat{\beta}_j - 0}{sd(\beta_j)} = \frac{\hat{\beta}_j}{\hat{\sigma} \sqrt{(X^\textsc{t}X)_{[j,j]}^{-1}}}
	\end{equation}
	
	When $\sigma^2$ is unknown, under $H_0$:
	
	\begin{equation}
	z_j \sim t_{\textsc{n}-p-1}
	\end{equation}
	
	where $t_k$ is a Student $t$ distribution with $k$ degrees of freedom
	
	When $\sigma^2$ is known, under $H_0$:
	
	\begin{equation}
	z_j \sim N(0;1)
	\end{equation}
	
	To test $H_0 : \beta_j, \beta_k = 0$:
	
	\begin{equation}
	F = \frac{(RSS_0 - RSS_1}{•}
	\end{equation}
	
	
	\subsection*{References}
	
	\bibliographystyle{ieeetr}
	\bibliography{bib}
	


	
\end{document}